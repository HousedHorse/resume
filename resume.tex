\documentclass[]{article}
\usepackage{lmodern}
\usepackage{amssymb,amsmath}
\usepackage{ifxetex,ifluatex}
\usepackage{fixltx2e} % provides \textsubscript
\ifnum 0\ifxetex 1\fi\ifluatex 1\fi=0 % if pdftex
  \usepackage[T1]{fontenc}
  \usepackage[utf8]{inputenc}
\else % if luatex or xelatex
  \ifxetex
    \usepackage{mathspec}
  \else
    \usepackage{fontspec}
  \fi
  \defaultfontfeatures{Ligatures=TeX,Scale=MatchLowercase}
\fi
% use upquote if available, for straight quotes in verbatim environments
\IfFileExists{upquote.sty}{\usepackage{upquote}}{}
% use microtype if available
\IfFileExists{microtype.sty}{%
\usepackage{microtype}
\UseMicrotypeSet[protrusion]{basicmath} % disable protrusion for tt fonts
}{}
\usepackage[margin=1in]{geometry}
\usepackage{hyperref}
\hypersetup{unicode=true,
            pdftitle={Résumé},
            pdfauthor={William Findlay},
            pdfborder={0 0 0},
            breaklinks=true}
\urlstyle{same}  % don't use monospace font for urls
\usepackage{biblatex}

\usepackage{graphicx,grffile}
\makeatletter
\def\maxwidth{\ifdim\Gin@nat@width>\linewidth\linewidth\else\Gin@nat@width\fi}
\def\maxheight{\ifdim\Gin@nat@height>\textheight\textheight\else\Gin@nat@height\fi}
\makeatother
% Scale images if necessary, so that they will not overflow the page
% margins by default, and it is still possible to overwrite the defaults
% using explicit options in \includegraphics[width, height, ...]{}
\setkeys{Gin}{width=\maxwidth,height=\maxheight,keepaspectratio}
\IfFileExists{parskip.sty}{%
\usepackage{parskip}
}{% else
\setlength{\parindent}{0pt}
\setlength{\parskip}{6pt plus 2pt minus 1pt}
}
\setlength{\emergencystretch}{3em}  % prevent overfull lines
\providecommand{\tightlist}{%
  \setlength{\itemsep}{0pt}\setlength{\parskip}{0pt}}
\setcounter{secnumdepth}{5}

%%% Use protect on footnotes to avoid problems with footnotes in titles
\let\rmarkdownfootnote\footnote%
\def\footnote{\protect\rmarkdownfootnote}

%%% Change title format to be more compact
\usepackage{titling}

% Create subtitle command for use in maketitle
\newcommand{\subtitle}[1]{
  \posttitle{
    \begin{center}\large#1\end{center}
    }
}

\setlength{\droptitle}{-2em}

  \title{Résumé}
    \pretitle{\vspace{\droptitle}\centering\huge}
  \posttitle{\par}
    \author{William Findlay}
    \preauthor{\centering\large\emph}
  \postauthor{\par}
      \predate{\centering\large\emph}
  \postdate{\par}
    \date{\today}

\usepackage{float}
\usepackage{listings}
\usepackage[hang,bf]{caption}
\usepackage{framed}
\usepackage[section]{placeins}

\usepackage[dvipsnames,table]{xcolor}

\definecolor{gray}{HTML}{606060}
\definecolor{black}{HTML}{000000}

\usepackage{pifont}

\allowdisplaybreaks

\usepackage[bottom]{footmisc}
% fancy headers/footers
\makeatletter
\usepackage{fancyhdr}
\lhead{}
\chead{}
\rhead{\vspace{1em}\color{gray}Last Updated: \today}
\lfoot{}
\cfoot{}
\rfoot{}
\renewcommand{\headrulewidth}{0pt}

\usepackage{amsmath, amsfonts,amssymb, amsthm}
\usepackage{siunitx}
\usepackage[boxruled,lined,linesnumbered,titlenumbered]{algorithm2e}

\usepackage{setspace}
\usepackage{changepage}
\usepackage[explicit]{titlesec}
\usepackage{aliascnt}

\floatplacement{figure}{!htbp}
\floatplacement{table}{!htbp}
\lstset{numbers=left,breaklines=true,frame=single,language=Python,captionpos=t,abovecaptionskip={\abovecaptionskip},
belowcaptionskip={0.5em},aboveskip=\intextsep,showstringspaces=false,identifierstyle=\color{Blue},
commentstyle={\color{OliveGreen}},keywordstyle={\bfseries\color{Orange}},stringstyle=\color{Purple},mathescape=true}
\setlength{\captionmargin}{1in}

\newgeometry{left=0.6in,top=1in,bottom=1in,right=0.6in}

\newtheoremstyle{plain}
{12pt}   % ABOVESPACE
{12pt}   % BELOWSPACE
{\itshape}  % BODYFONT
{0pt}       % INDENT (empty value is the same as 0pt)
{\bfseries} % HEADFONT
{.}         % HEADPUNCT
{5pt plus 1pt minus 1pt} % HEADSPACE
{}          % CUSTOM-HEAD-SPEC

\newtheoremstyle{definition}
{12pt}   % ABOVESPACE
{12pt}   % BELOWSPACE
{\normalfont}  % BODYFONT
{0pt}       % INDENT (empty value is the same as 0pt)
{\bfseries} % HEADFONT
{.}         % HEADPUNCT
{5pt plus 1pt minus 1pt} % HEADSPACE
{}          % CUSTOM-HEAD-SPEC

\newtheoremstyle{remark}
{12pt}   % ABOVESPACE
{12pt}   % BELOWSPACE
{\normalfont}  % BODYFONT
{0pt}       % INDENT (empty value is the same as 0pt)
{\itshape} % HEADFONT
{.}         % HEADPUNCT
{5pt plus 1pt minus 1pt} % HEADSPACE
{}          % CUSTOM-HEAD-SPEC

\theoremstyle{plain}

% define theorem
\newtheorem{theorem}{Theorem}[section]
\providecommand*{\theoremautorefname}{Theorem}

% define lemma
\newtheorem{lemma}{Lemma}[section]
\providecommand*{\lemmaautorefname}{Lemma}

% define claim
\newtheorem{claim}{Claim}[section]
\providecommand*{\claimautorefname}{Claim}

% define corollary
\newtheorem{corollary}{Corollary}[section]
\providecommand*{\corollaryautorefname}{Corollary}

% define proposition
\newtheorem{proposition}{Proposition}[section]
\providecommand*{\propositionautorefname}{Proposition}

% define conjecture
\newtheorem{conjecture}{Conjecture}[section]
\providecommand*{\conjectureautorefname}{Conjecture}

\theoremstyle{remark}

% define observation
\newtheorem{observation}{Observation}[section]
\providecommand*{\observationautorefname}{Observation}

% define remark
\newtheorem{remark}{Remark}[section]
\providecommand*{\remarkautorefname}{Remark}

\theoremstyle{definition}

% define example
\newtheorem{example}{Example}[section]
\providecommand*{\exampleautorefname}{Example}

% define definition
\newtheorem{definition}{Definition}[section]
\providecommand*{\definitionautorefname}{Definition}

\newcommand{\blackbox}{\hfill$\blacksquare$}
\usepackage{tikz}
\newcommand*\circled[1]{\tikz[baseline=(char.base)]{
            \node[shape=circle,draw,inner sep=2pt] (char) {#1};}}

\renewcommand{\labelitemi}{$\bullet$}
\renewcommand{\labelitemii}{\ding{226}}
\renewcommand{\labelitemiii}{\tiny$\blacksquare$}
\renewcommand{\labelitemiv}{\small$\triangleright$}

\titleformat{\paragraph} % command to change
[runin]                  % shape  (runin, etc.)
{\bfseries}              % format (bfseries, itshape, etc.)
{}                       % label  (thesection, thesubsection, etc.)
{0em}                    % separation between label and body
{#1}        % before the body
[.]                       % after the body

\titleformat{\subparagraph} % command to change
[runin]                  % shape  (runin, etc.)
{\itshape}              % format (bfseries, itshape, etc.)
{}                       % label  (thesection, thesubsection, etc.)
{0em}                    % separation between label and body
{#1}        % before the body
[.]                       % after the body

\let\lil\lstinputlisting
\usepackage{afterpage}
\hypersetup{colorlinks, allcolors=., urlcolor=blue}

\usepackage{etoolbox}% http://ctan.org/pkg/etoolbox
\makeatletter
\patchcmd{\lst@GLI@}% <command>
  {\def\lst@firstline{#1\relax}}% <search>
  {\def\lst@firstline{#1\relax}\def\lst@firstnumber{#1\relax}}% <replace>
  {\typeout{listings firstnumber=firstline}}% <success>
  {\typeout{listings firstnumber not set}}% <failure>
\makeatother

\renewcommand\lstlistlistingname{List of Listings}
\usepackage{chngcntr}
\counterwithin{figure}{section}
\counterwithin{table}{section}

\usepackage{booktabs}
\usepackage{longtable}
\usepackage{array}
\usepackage{multirow}
\usepackage{wrapfig}
\usepackage{float}
\usepackage{colortbl}
\usepackage{pdflscape}
\usepackage{tabu}
\usepackage{threeparttable}
\usepackage{threeparttablex}
\usepackage[normalem]{ulem}
\usepackage{makecell}
\pagestyle{fancy}

\renewcommand{\sectionautorefname}{Section}
\renewcommand{\subsectionautorefname}{Subection}
\renewcommand{\subsubsectionautorefname}{Subection}
\renewcommand{\paragraphautorefname}{}
\renewcommand{\subparagraphautorefname}{}

\usepackage{nameref}

\makeatletter
\renewcommand{\theparagraph}{\bfseries \@currentlabelname}
\renewcommand{\thesubparagraph}{\itshape \@currentlabelname}
\makeatother

\definecolor{purple}{HTML}{400080}

\titleformat{\section}
{\color{black}\Large\scshape}
{}
{0em}
{#1}[{\color{gray}\hrule}]

\titleformat{\subsection}
{\color{black}\bfseries}
{}
{0em}
{#1}

\titleformat{\subsubsection}
{\color{black}}
{}
{0em}
{#1}[]

\titlespacing{\section}
{0em}{1em}{0.5em}

\titlespacing{\subsection}
{0em}{0.5em}{0em}

\titlespacing{\subsubsection}
{0em}{0.5em}{0em}

\renewcommand{\maketitle}{\relax}

\usepackage{enumitem}
\setitemize{itemsep=0em}

\newcommand{\sep}{
  {\color{black}$\bullet$}
}

\begin{document}
\maketitle

\begin{figure}
\small
\begin{minipage}[t]{0.65\textwidth}
\begin{center}
{\Huge\bfseries {\color{gray}William} Findlay}

\color{gray}

\href{https://www.github.com/willfindlay}{\color{gray}github.com/willfindlay} $\bullet$ \href{http://www.wfindlay.com}{\color{gray}wfindlay.com}\\
{\color{gray}(613) 296-1240} $\bullet$ \href{mailto:william@wfindlay.com}{\color{gray}william@wfindlay.com}
\end{center}

\color{gray}

\section{Education}
\color{black}
\textbf{Bachelor of Computer Science} \hfill \emph{Carleton University}\\
{September 2015 - April 2020} \hfill\emph {Ottawa, ON}
\color{gray}

Computer and Network Security Stream\\
Accelerated Masters Program\\
CGPA: 11.0 (A)

\section{Work Experience}

\color{black}
\textbf{Undergraduate Researcher} (Linux Kernel Security) \hfill \emph{Carleton University}\\
April 2019 - Present \hfill\emph {Ottawa, ON}
\color{gray}
\begin{itemize}[itemsep=0em]
\item Researching the viability of eBPF-based implementations for intrusion detection systems on the GNU/Linux operating system.
\item Designed and implemented \texttt{ebpH}, an intrusion detection system that uses eBPF-based observability filters to establish regular process behavior and detect anomalous system call patterns
\item Project is currently closed-source pending publication.
\end{itemize}

\color{black}
\textbf{Teaching Assistant} (COMP3000 Operating Systems) \hfill \emph{Carleton University}\\
September 2018 - Present \hfill\emph {Ottawa, ON}
\color{gray}
\begin{itemize}[itemsep=0em]
\item Nominee for the \href{https://carleton.ca/tasupport/taawards/edc-outstanding-ta-awards/}{Outstanding Teaching Assistant Award}
\item Ran tutorial sessions for groups of 50 students.
\item Took a leadership role to ensure tutorials proceeded smoothly.
\item Held weekly office hours and workshops for students.
\item Graded assignments and tests and gave appropriate feedback.
\item Proctored exams for about 200 students.
\item Assisted professor with development and delivery of course material.
\end{itemize}

\color{black}
\textbf{Service Department Supervisor} \hfill \emph{Metro Ontario, Inc.}\\
April 2014 - January 2018 \hfill\emph {Ottawa, ON}
\color{gray}
\begin{itemize}[itemsep=0em]
\item Managed day-to-day operations in the front end service department.
\item In charge of store payroll and accounting on a part-time basis.
\item Exhibited superior customer service skills as required.
\end{itemize}

\section{Projects \normalfont \small (See more on \href{https://www.github.com/willfindlay}{GitHub})}

\subsection{Raspberry Pi GPIO Pin Driver}
\begin{itemize}[itemsep=0em]
\item Wrote a character device driver Linux kernel module to act as a driver for GPIO pins on the Raspberry Pi.
\item Module supported read and write operations to set active pins, read pin input, and write pin output.
\item A full writeup on the module is available at \url{https://wfindlay.com/assets/written/3000report.pdf}
\end{itemize}

\subsection{Asciify (ASCII Art Generator)}
\begin{itemize}[itemsep=0em]
\item Wrote an ASCII art generator CLI in Python3.
\item Generator takes an image or GIF as input and produces ASCII art output.
\item Full support for color and black and white ASCII art.
\item Full support for ASCII animated GIFs.
\end{itemize}

\subsection{Genetic Algorithm for Text Generation}
\begin{itemize}[itemsep=0em]
\item Wrote a genetic algorithm in C++ to generate a human-readable string from random data.
\end{itemize}

\end{minipage}
\hfill
\begin{minipage}[t]{0.32\textwidth}
\color{gray}
\ %
\vspace{-1.5em}

\section{Languages}

\subsection{Programming}

\subsubsection{10,000 or more lines}

C, C++, Python

\subsubsection{5,000 - 10,000 lines}

Java, Javascript

\subsubsection{1,000 - 5,000 lines}

Haskell, Prolog, Vimscript, R, Common Lisp

\subsection{Markup}

Markdown, Rmarkdown, \LaTeX, HTML, CSS

\subsection{Human}

English, French

\section{Technologies}

Linux Kernel, eBPF, bcc, Qt, NodeJS, Git, Bash

\section{Workflow}

\subsection{Operating System}
GNU/Linux (Arch Linux)

\subsection{Window Manager}
i3wm

\subsection{Terminal and Shell}
Termite, Bash

\subsection{Text Editor}
Vim

\subsection{Version Control}
Git, GitHub

\section{Technical Skills}
\begin{itemize}[itemsep=0em]
\item GNU/Linux kernel development
\item Low-level C development
\item C++ development
\item Python scripting
\item Statistical analysis in Python (pandas, numpy, scipy) and R
\item Documentation in \LaTeX{}
\end{itemize}

\section{General Skills}
\begin{itemize}[itemsep=0em]
\item Team leader
\item Dedicated
\item Goal-oriented
\end{itemize}

\section{Extracurricular}
\begin{itemize}[itemsep=0em]
\item Avid Linux user
\item Free software enthusiast
\item Computer builder
\item Study group leader for friends and peers
\end{itemize}

\end{minipage}
\end{figure}

\printbibliography


\end{document}
